\documentclass{beamer}

\usetheme{default}

\setbeamertemplate{navigation symbols}{}

\usepackage{polyglossia}

\setmainlanguage[variant=usmax]{english}
\setotherlanguage{spanish}

\usepackage{fancyvrb}

\DefineShortVerb{\|}
\DefineVerbatimEnvironment{code}{Verbatim}{frame=lines}

\title{Introducción a Haskell}
\author{Juan Pedro Villa Isaza}
\institute{Stack Builders}
\date{27 de noviembre de 2014}

\logo{\includegraphics[height=0.5cm]{stackbuilders.png}}

\begin{document}

%%%%%%%%%%%%%%%%%%%%%%%%%%%%%%%%%%%%%%%%%%%%%%%%%%%%%%%%%%%%%%%%%%%%%%%%%%%%%%

\frame{\titlepage}

%%%%%%%%%%%%%%%%%%%%%%%%%%%%%%%%%%%%%%%%%%%%%%%%%%%%%%%%%%%%%%%%%%%%%%%%%%%%%%

\begin{frame}
  \frametitle{Introducción a Haskell}

  \begin{center}
    \url{http://www.haskell.org/}
  \end{center}
\end{frame}

%%%%%%%%%%%%%%%%%%%%%%%%%%%%%%%%%%%%%%%%%%%%%%%%%%%%%%%%%%%%%%%%%%%%%%%%%%%%%%

\begin{frame}[fragile]
  \frametitle{Introducción a Haskell}
  \framesubtitle{¡Hola, mundo!}

  \begin{itemize}
  \item
    \begin{code}
$ cat hello-world.hs
main :: IO ()
main = putStrLn "¡Hola, mundo!"
    \end{code}
  \item
    \begin{code}
$ ghc hello-world.hs
...
$ ./hello-world
¡Hola, mundo!
    \end{code}
  \end{itemize}
\end{frame}

%%%%%%%%%%%%%%%%%%%%%%%%%%%%%%%%%%%%%%%%%%%%%%%%%%%%%%%%%%%%%%%%%%%%%%%%%%%%%%

\begin{frame}[fragile]
  \frametitle{Introducción a Haskell}
  \framesubtitle{¿Hola, mundo?}

  \begin{itemize}
  \item
    \begin{code}
$ cat JustHaskell.hs
...
factorial :: Integer -> Integer
factorial 0 = 1
factorial n = n * factorial (n - 1)
...
    \end{code}
  \item
    \begin{code}
$ ghci JustHaskell.hs
...
JustHaskell> factorial 5
120
    \end{code}
  \end{itemize}
\end{frame}

%%%%%%%%%%%%%%%%%%%%%%%%%%%%%%%%%%%%%%%%%%%%%%%%%%%%%%%%%%%%%%%%%%%%%%%%%%%%%%

\begin{frame}[fragile]
  \frametitle{Introducción a Haskell}
  \framesubtitle{Quicksort}

  \scriptsize
  \begin{code}
qsort :: Ord a => [a] -> [a]
qsort []     = []
qsort (x:xs) =
  qsort (filter (<= x) xs) ++ [x] ++ qsort (filter (> x) xs)
  \end{code}
  \normalsize
\end{frame}

%%%%%%%%%%%%%%%%%%%%%%%%%%%%%%%%%%%%%%%%%%%%%%%%%%%%%%%%%%%%%%%%%%%%%%%%%%%%%%

\begin{frame}[fragile]
  \frametitle{Introducción a Haskell}
  \framesubtitle{Programación funcional pura}

  \begin{itemize}
  \item
    \begin{code}
x :: Int
x = 1
    \end{code}
  \item
    \begin{code}
x = 2
    \end{code}
  \item
    \begin{code}
    Multiple declarations of ‘x’...
    \end{code}
  \end{itemize}
\end{frame}

%%%%%%%%%%%%%%%%%%%%%%%%%%%%%%%%%%%%%%%%%%%%%%%%%%%%%%%%%%%%%%%%%%%%%%%%%%%%%%

\begin{frame}[fragile]
  \frametitle{Introducción a Haskell}
  \framesubtitle{Programación funcional pura}

  \begin{itemize}
  \item
    \small
    \begin{code}
fibonacci :: Integer -> Integer
fibonacci 0 = 0
fibonacci 1 = 1
fibonacci n = fibonacci (n - 1) + fibonacci (n - 2)
    \end{code}
    \normalsize
  \item
    \begin{code}
JustHaskell> fibonacci 5
5
    \end{code}
  \end{itemize}
\end{frame}

%%%%%%%%%%%%%%%%%%%%%%%%%%%%%%%%%%%%%%%%%%%%%%%%%%%%%%%%%%%%%%%%%%%%%%%%%%%%%%

\begin{frame}[fragile]
  \frametitle{Introducción a Haskell}
  \framesubtitle{Tipado estático}

  \begin{itemize}
  \item
    \begin{code}
JustHaskell> let x = 1 :: Int
JustHaskell> let y = 2 :: Double
JustHaskell> x + y
    Couldn't match expected type...
    \end{code}
  \item
    \begin{code}
JustHaskell> False && "True"
    Couldn't match expected type...
    \end{code}
  \end{itemize}
\end{frame}

%%%%%%%%%%%%%%%%%%%%%%%%%%%%%%%%%%%%%%%%%%%%%%%%%%%%%%%%%%%%%%%%%%%%%%%%%%%%%%

\begin{frame}[fragile]
  \frametitle{Introducción a Haskell}
  \framesubtitle{Inferencia de tipos}

  \begin{itemize}
  \item
    \begin{code}
data [] a = [] | a : [a]
    \end{code}
  \item
    \begin{code}
sum []     = 0
sum (x:xs) = x + sum xs
    \end{code}
  \item
    \begin{code}
JustHaskell> :t sum
sum :: Num a => [a] -> a
    \end{code}
  \end{itemize}
\end{frame}

%%%%%%%%%%%%%%%%%%%%%%%%%%%%%%%%%%%%%%%%%%%%%%%%%%%%%%%%%%%%%%%%%%%%%%%%%%%%%%

\begin{frame}[fragile]
  \frametitle{Introducción a Haskell}
  \framesubtitle{Evaluación perezosa}

  \begin{itemize}
  \item
    \begin{code}
JustHaskell> take 10 [1..]
[1,2,3,4,5,6,7,8,9,10]
    \end{code}
  \item
    %% Taken from Real World Haskell (p. xxiv).
    \begin{code}
JustHaskell> let minima n xs = take n (sort xs)
JustHaskell> minima 10 [100,99..1]
[1,2,3,4,5,6,7,8,9,10]
    \end{code}
  \end{itemize}
\end{frame}

%%%%%%%%%%%%%%%%%%%%%%%%%%%%%%%%%%%%%%%%%%%%%%%%%%%%%%%%%%%%%%%%%%%%%%%%%%%%%%

\begin{frame}[fragile]
  \frametitle{Introducción a Haskell}
  \framesubtitle{Programación \emph{wholemeal}}

  %% Taken from Brent Yorgey's Introduction to Haskell (Haskell
  %% Basics).
  \begin{itemize}
  \item[Java:]
    \small
    \begin{code}
int[] xs = {2,3,5,7,11};

int total = 0;
for (int i = 0; i < xs.length; i++) {
  total = total + 3 * xs[i];
}

System.out.println(total);
    \end{code}
    \normalsize
  \item[Haskell:]
    \small
    \begin{code}
xs = [2,3,5,7,11]

total = sum (map (3*) xs)

main = print total
    \end{code}
    \normalsize
  \end{itemize}
\end{frame}

%%%%%%%%%%%%%%%%%%%%%%%%%%%%%%%%%%%%%%%%%%%%%%%%%%%%%%%%%%%%%%%%%%%%%%%%%%%%%%

\begin{frame}
  \frametitle{Introducción a Haskell}
  \framesubtitle{Comunidad}

  \begin{itemize}
  \item
    ``Un esfuerzo de más de 20 años por parte de la comunidad de
    código abierto''.
  \item
    Wiki.
  \item
    Canales de IRC, listas de correo.
  \item
    Google+, Planet Haskell, Reddit, Stack Overflow.
  \item
    Más de 6950 paquetes.
  \item
    ...
  \end{itemize}
  \vfill
  \begin{thebibliography}{Haskell}
  \bibitem[Haskell]{haskell}
    \url{http://www.haskell.org/}
  \end{thebibliography}
\end{frame}

%%%%%%%%%%%%%%%%%%%%%%%%%%%%%%%%%%%%%%%%%%%%%%%%%%%%%%%%%%%%%%%%%%%%%%%%%%%%%%

\begin{frame}
  \frametitle{Introducción a Haskell}
  \framesubtitle{Bibliografía}

  \begin{thebibliography}{O'Sullivan, Goerzen, and Stewart 2008}
  \setbeamertemplate{bibliography item}[book]
  \bibitem[Bird 2014]{bird-2014}
    Bird, Richard (2014).
    \newblock \emph{Thinking Functionally with Haskell}.
    \newblock Cambridge University Press.
  \setbeamertemplate{bibliography item}[book]
  \bibitem[Lipovača 2011]{lipovača-2011}
    Lipovača, Miran (2011).
    \newblock \emph{Learn You a Haskell for Great Good! A Beginner's Guide}.
    \newblock No Starch Press.
  \setbeamertemplate{bibliography item}[book]
  \bibitem[O'Sullivan, Goerzen, and Stewart 2008]{osullivan-2008}
    O'Sullivan, Bryan, John Goerzen, and Don Stewart (2008).
    \newblock \emph{Real World Haskell}.
    \newblock O'Reilly.
  \end{thebibliography}
\end{frame}

%%%%%%%%%%%%%%%%%%%%%%%%%%%%%%%%%%%%%%%%%%%%%%%%%%%%%%%%%%%%%%%%%%%%%%%%%%%%%%

\end{document}
