\documentclass{beamer}

\usetheme{default}

\setbeamertemplate{navigation symbols}{}

\usepackage{polyglossia}

\setmainlanguage[variant=usmax]{english}
\setotherlanguage{spanish}

\usepackage{fancyvrb}

\DefineShortVerb{\|}
\DefineVerbatimEnvironment{code}{Verbatim}{frame=lines}

\title{Haskell}
\subtitle{An advanced purely-functional programming language}
\author{}
\institute{Stack Builders}
\date{November 27, 2014}

\logo{\includegraphics[height=0.5cm]{stackbuilders.png}}

\begin{document}

%%%%%%%%%%%%%%%%%%%%%%%%%%%%%%%%%%%%%%%%%%%%%%%%%%%%%%%%%%%%%%%%%%%%%%%%%%%%%%

\frame{\titlepage}

%%%%%%%%%%%%%%%%%%%%%%%%%%%%%%%%%%%%%%%%%%%%%%%%%%%%%%%%%%%%%%%%%%%%%%%%%%%%%%

\begin{frame}
  \frametitle{Haskell}

  \begin{center}
    \url{http://www.haskell.org}
  \end{center}
\end{frame}

%%%%%%%%%%%%%%%%%%%%%%%%%%%%%%%%%%%%%%%%%%%%%%%%%%%%%%%%%%%%%%%%%%%%%%%%%%%%%%

\begin{frame}[fragile]
  \frametitle{Haskell}
  \framesubtitle{Hello, World!}

  \begin{itemize}
  \item
    \begin{code}
$ cat hello-world.hs
main :: IO ()
main = putStrLn "Hello"
    \end{code}
  \item
    \begin{code}
$ ghc hello-world.hs
...
$ ./hello-world
Hello, World!
    \end{code}
  \end{itemize}
\end{frame}

%%%%%%%%%%%%%%%%%%%%%%%%%%%%%%%%%%%%%%%%%%%%%%%%%%%%%%%%%%%%%%%%%%%%%%%%%%%%%%

\begin{frame}[fragile]
  \frametitle{Haskell}
  \framesubtitle{Hello, World?}

  \begin{itemize}
  \item
    \begin{code}
$ cat JustHaskell.hs
...
factorial :: Integer -> Integer
factorial 0 = 1
factorial n = n * factorial (n - 1)
    \end{code}
  \item
    \begin{code}
$ ghci JustHaskell.hs
JustHaskell> factorial 5
120
    \end{code}
  \end{itemize}
\end{frame}

\begin{frame}[fragile]
  \frametitle{Haskell}
  \framesubtitle{Quicksort}

  \scriptsize
  \begin{code}
qsort :: Ord a => [a] -> [a]
qsort []     = []
qsort (x:xs) =
  qsort (filter (<= x) xs) ++ [x] ++ qsort (filter (> x) xs)
  \end{code}
  \normalsize
\end{frame}

%%%%%%%%%%%%%%%%%%%%%%%%%%%%%%%%%%%%%%%%%%%%%%%%%%%%%%%%%%%%%%%%%%%%%%%%%%%%%%

\begin{frame}[fragile]
  \frametitle{Haskell}
  \framesubtitle{Features}

  \begin{block}{Statically typed}
    \begin{itemize}
    \item
      \begin{code}
JustHaskell> let x = 1 :: Int
JustHaskell> let y = 2 :: Double
JustHaskell> x + y
    Couldn't match expected type...
      \end{code}
    \item
      \begin{code}
JustHaskell> False && "True"
    Couldn't match expected type...
      \end{code}
    \end{itemize}
  \end{block}
\end{frame}

%%%%%%%%%%%%%%%%%%%%%%%%%%%%%%%%%%%%%%%%%%%%%%%%%%%%%%%%%%%%%%%%%%%%%%%%%%%%%%

\begin{frame}[fragile]
  \frametitle{Haskell}
  \framesubtitle{Features}

  \begin{block}{Purely functional}
    \begin{itemize}
    \item
      \small
      \begin{code}
fibonacci :: Integer -> Integer
fibonacci 0 = 0
fibonacci 1 = 1
fibonacci n = fibonacci (n - 1) + fibonacci (n - 2)
      \end{code}
      \normalsize
    \end{itemize}
  \end{block}
\end{frame}

%%%%%%%%%%%%%%%%%%%%%%%%%%%%%%%%%%%%%%%%%%%%%%%%%%%%%%%%%%%%%%%%%%%%%%%%%%%%%%

\begin{frame}[fragile]
  \frametitle{Haskell}
  \framesubtitle{Features}

  \begin{block}{Type inference}
    \begin{itemize}
    \item
      \begin{code}
data [] a = [] | a : [a]
      \end{code}
    \item
      \begin{code}
$ cat JustHaskell.hs
...
sum []     = 0
sum (x:xs) = x + sum xs
      \end{code}
    \item
      \begin{code}
$ ghci JustHaskell.hs
JustHaskell> :t sum
sum :: Num a => [a] -> a
      \end{code}
    \end{itemize}
  \end{block}
\end{frame}

%%%%%%%%%%%%%%%%%%%%%%%%%%%%%%%%%%%%%%%%%%%%%%%%%%%%%%%%%%%%%%%%%%%%%%%%%%%%%%

\begin{frame}[fragile]
  \frametitle{Haskell}
  \framesubtitle{Features}

  \begin{block}{Laziness}
    \begin{itemize}
    \item
      \begin{code}
JustHaskell> take 10 [1..]
[1,2,3,4,5,6,7,8,9,10]
      \end{code}
    \item
      \begin{code}
JustHaskell> take 3 (repeat 7)
[7,7,7]
      \end{code}
    \end{itemize}
  \end{block}
\end{frame}

%%%%%%%%%%%%%%%%%%%%%%%%%%%%%%%%%%%%%%%%%%%%%%%%%%%%%%%%%%%%%%%%%%%%%%%%%%%%%%

\begin{frame}
  \frametitle{Haskell}
  \framesubtitle{Community}

  \begin{itemize}
  \item
    ``An open source community effort for over 20 years.''
  \item
    ``The Haskell community is spread out online across several mediums and
    around the world!''
    \begin{itemize}
    \item Mailing lists, IRC channels, StackOverflow, Google+, Reddit,
      the Haskell wiki, Planet Haskell
    \end{itemize}
  \item
    Open source contribution to Haskell is very active.
    \begin{itemize}
    \item
      Over 6950 packages freely available.
    \end{itemize}
  \end{itemize}
  \vfill
  \begin{thebibliography}{Haskell}
  \bibitem[Haskell]{haskell}
    \url{http://www.haskell.org/}
  \end{thebibliography}
\end{frame}

%%%%%%%%%%%%%%%%%%%%%%%%%%%%%%%%%%%%%%%%%%%%%%%%%%%%%%%%%%%%%%%%%%%%%%%%%%%%%%

\begin{frame}
  \frametitle{Haskell}
  \framesubtitle{Documentation}

  \begin{thebibliography}{O'Sullivan, Goerzen, and Stewart 2008}
  \setbeamertemplate{bibliography item}[book]
  \bibitem[Bird 2014]{bird-2014}
    Bird, Richard (2014).
    \newblock \emph{Thinking Functionally with Haskell}.
    \newblock Cambridge University Press.
  \setbeamertemplate{bibliography item}[book]
  \bibitem[Lipovača 2011]{lipovača-2011}
    Lipovača, Miran (2011).
    \newblock \emph{Learn You a Haskell for Great Good! A Beginner's Guide}.
    \newblock No Starch Press.
  \setbeamertemplate{bibliography item}[book]
  \bibitem[O'Sullivan, Goerzen, and Stewart 2008]{osullivan-2008}
    O'Sullivan, Bryan, John Goerzen, and Don Stewart (2008).
    \newblock \emph{Real World Haskell}.
    \newblock O'Reilly.
  \end{thebibliography}
\end{frame}

%%%%%%%%%%%%%%%%%%%%%%%%%%%%%%%%%%%%%%%%%%%%%%%%%%%%%%%%%%%%%%%%%%%%%%%%%%%%%%

\end{document}
